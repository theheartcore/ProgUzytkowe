\documentclass[a4paper,11pt]{article}
\usepackage{enumerate}
\usepackage{polski}
\usepackage[utf8]{inputenc}
\usepackage[pdftex]{hyperref}
\usepackage{makeidx}
\usepackage[tableposition=top]{caption}
\usepackage{algorithmic}
\usepackage{graphicx}
\usepackage{enumerate}
\usepackage{multirow}
\usepackage{amsmath}
\usepackage{amssymb} 
\usepackage[table]{xcolor}
\begin{document}

\begin{tabular}{r|r|c} \hline
\hline
$x_{1}$ & $x_{2}$ & $(x_{1}AND x_{2})$ \\ \hline
1 & 1 & 1\\
1 & 0 & 0\\
0 & 1 & 0\\
0 & 0 & 0 \\
\hline \hline
\end{tabular}
\newpage

\begin{tabular}{|r|l|} \hline
7C0 & heksadecymalnie \\
3700 & oktalnie \\
11111000000 & binarnie \\
\hline \hline
1984 & dziesietnie \\ \hline
\end{tabular}

\newpage
\begin{tabular}{l|c|r}
  \hline
 Niektore & \cellcolor{red}Pokolorowane & Komory \\
  \hline
	\end{tabular}
	\newpage
	
\begin{table}
\begin{center}
\begin{tabular}{|c|c|c|c|c|c|}
\cline{5-5}
\multicolumn{4}{c|}{}{}{} & Primes \\
\hline
A & A1 & A2 & A3 \\
\hline
B & B1 & B2 & B3 \\
\hline
C & C1 & C2 & C3 \\
\hline
\end{tabular}
\end{center}
\caption{Table caption.}
\label{Table1}
\end{table}

	

\end{document}