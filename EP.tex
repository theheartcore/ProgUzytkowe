\documentclass{article}
\usepackage[a4paper,left=3.5cm,right=2.5cm,top=2.5cm,bottom=2.5cm]{geometry}
%%\usepackage[MeX]{polski}
%%\usepackage[cp1250]{inputenc}
\usepackage{polski}
\usepackage[utf8]{inputenc}
\usepackage[pdftex]{hyperref}
\usepackage{makeidx}
\usepackage[tableposition=top]{caption}
\usepackage{algorithmic}
\usepackage{graphicx}
\usepackage{enumerate}
\usepackage{multirow}
\usepackage{amsmath} %pakiet matematyczny
\usepackage{amssymb} %pakiet dodatkowych symboli
\begin{document}
Tu umieszczamy kod TeXa, ktory bedzie kompilowany,
żłóąśęćźżż.
\begin {document}
\tableofcontents
\section{podtytul}
Tu umieszczamy kod TeXa, ktroy bedzie kompilowany,
Kamapkaść. Pozycja \cite{dote:ks} mowi o

Powierzchnia $m^2$ dlua $u_1$

\begin{itemize}
\item podpunkt
\begin{itemize}
	\item podpunkty
	\item podpunkty
\end{itemize}
\end{itemize}

\begin{enumerate}[I)]
	\item ffdsadasd
	
\end{enumerate}
\begin{description}
\item [itemize] to środowisko do wypunktowania
\item [enumerate] to środowisko do numerowania punktów
\end{description}
fi ff ft
\newline
\newline
\newline
\newline
asdfsdadsdasdasd


Kamaaaaapka
\begin{document}
\left[
\begin{array{cccc}
a_{11} & a_{12} & \ldots & a_{1K} \\
a_{21} & a_{22} & \ldots & a_{1K} \\
\vdots & \vdots & \ddots & \vdots \\
a_{K1} & a_{K2} & \ldots & a_{KK} \\
\end{array}
\right]
\end{equation}
\end{document}
